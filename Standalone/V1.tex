\documentclass{article}
\usepackage{graphicx} % Required for inserting images

\title{Standalone}
\author{Ronald Colyar}
\date{August 2024}

\begin{document}

\maketitle

\section{The Problem}
2019 Dodge Charger Scatpack 6.4L HEMI - It has 4 main computers that are required for the engine to begin to run in its stock configuration.

\begin{itemize}
    \item RFHub
    \item PCM
    \item BCM
    \item TCM
\end{itemize}
These computers can be a headache for those who want to attempt engine swaps or simple engine tuning, we often hear older non-technical people in the car world mention that back in their day it wasn't so complicated to get things running.
\vspace{0.5cm}

\noindent
These four computers must be from the same vehicle or they must have the same VIN programming in order for them to work and it should \textbf{never} work this way!
\vspace{0.5cm}

\noindent
Ideally an individual should be able to have one simple ECU to run their engine with an appropriate set of performance features without paying thousands of dollars.

\newpage
\section{The Target Features}

\begin{itemize}
    \item Display Screen(?)
    \item Real Time Data Logging(Wireless?)
    \item Modern tuning interface
    \item Wheelie control
    \item Real Time Speed Density Based Tuning
    \item Extremely custom engine protection
    \item Custom Warnings
    \item Launch Control
    \item Traction Control
    \item AI Apprentice(?)
    \item Race Application Sensor Support
    \item 8 x Ignition Outputs
    \item 8 x Fuel Injection Outputs
    \item Drive By Wire (DBW) Throttle Control
    \item Anti-Lag
    \item Tune by gear
    \item Multi-fuel Support- Support for Petrol (Gasoline), Methanol and Ethanol
    
\end{itemize}

\section{AI Apprentice}
Create multiple categories of surface+weather types and use your revisions along with data from the logs to train your apprentice who will eventually get just as good as you at finding a near perfect tune for the surface+weather combo in question.

\vspace{0.5cm}

\noindent
I think this feature will be game changing for racers who would prefer to use their previous tuning sessions to automatically optimize their car for the surface in question. One test pull and your apprentice should be able to dial your car in for the next run. This is ideal for no prep racers.

\section{No CAN or OBD2 compliance}
There will be no CAN system on board or any clearable DTCs due to the purpose of this system. There will be clearable warnings and Errors.
\newpage
\section{Technical Decisions}
I would like to document the technical decisions of this project step by step from the start to end.
\vspace{0.5cm}

\noindent
\textbf{The VE/Ignition/AFR Memory Structure}
\vspace{0.5cm}

VE Table =  RPM $\rightarrow$ MAP $\rightarrow$ VE percentage

Boost By gear Table = GEAR $\rightarrow$  MAX BOOST PSI(MAP)

AFR Table = RPM $\rightarrow$ MAP $\rightarrow$ Target AFR

Tune By Gear = GEAR $\rightarrow$ VE Table

\vspace{0.5cm}
[RPM(s)...MAP(s)...VE Percentages for each rpm(arrays positioned in memory by rpm index and indexed by current MAP)]

\vspace{0.5cm}
So given we have two rpms [1000,1100] and we defaulted all of the VE percentages to 50\% for 1000 and 40\% for 1100 and we only have three map values [10,20,30], our memory would look this [[1000,1100],[10,20,30], [50,50,50], [40,40,40]].


\vspace{0.5cm}
So the algorithm to get the correct VE while the engine is running would go like this:
\begin{enumerate}
    \item Gather the RPM in our RPM array we are closest to
    \item Use that RPM to generate a reference to the correct VE percentages array
    \item Gather the MAP in our MAP array we are closest to
    \item Take the index of the closest MAP and index the VE percentages array
\end{enumerate}

\vspace{0.5cm}
So lets walk through an example of this algorithm running in practice... We have this data [[1000,1100],[10,20,30], [51,52,53], [41,42,43]]. 

\vspace{0.5cm}
RPM = 800, MAP = 10(psi)
\begin{enumerate}
    \item closestRpm = 1000
    \item closestRPMArray = [51,52,53]
    \item closestMap = 10 , closestMapIndex = 0
    \item VEPercentage = closestRPMArray[closestMapIndex] = 51\%
\end{enumerate} 

This is just something I came up with as a first solution, I am certain we can optimize this algorithm a great deal by using hashing techniques, but the goal here is a working MVP before any optimizations.
\end{document}
